\documentclass{article}

\title{Does environmental variability help or harm population growth?}

\begin{document}
  \maketitle

\section{Group meeting, February 14, 2019}
We got updates and new goals for the pipeline and environmental covariate groups.
\subsection{Pipeline group}
\begin{itemize}
\item{Bones of the pipeline code have come together}
\item{Tom had some computer compatibility issues that may have something to do with popler version and package updates. Tom will troubleshoot this.}
\item{We still need an effective criterion for dropping species with sparse data. Tom tried filtering out species that were unobserved in at least one year. This helps but still includes species with very wide lambda posteriors, so we will need to do additional filtering.}
\item{Right now there are both JAGS and Stan versions of the pipeline. We need to pick one and run with it, and this may depend on how we choose to handle the splines.}
\item{The main pipeline work ahead is to develop code for the spline analysis. This is a big goal ahead of our next meeting.}
\item{We cannot estimate $\lambda$ as a stochastic node and then fit $\lambda$ to covariate all in the same model. We will need a strategy for dealing with this.}
\item{We will need to decide whether or not to include experimental studies. We can pull out just the control data but this may be tedious for some studies. We also need a way to aggregate structured data. Aldo will need to help with this.}
\end{itemize}

\subsection{Covariate group}
\begin{itemize}
\item{The group made great progress (including nice markdown docs!) on pulling historical climate data for the LTER site locations, as well as future climate projections.}
\item{So far, we have monthly temp and precip data. The next step will be to convert this to SPEI. We need to figure out the temporal resolution required for this. Can we calculate SPEI from monthly data or do we need finer temporal scale?}
\item{We need to align the climate year with the demographic year, and this will require information on when (what month) the observations in each data set were collected. Unfortunately this information is not conveniently provided by popler (though sometimes it is in the covariates). We will need to collect this data manually, and Tom has started a google sheet for recording this, for all the observational studies that are not individual or basal area.}
\end{itemize}

\section{Group meeting, January 24, 2019}
Here is a run-down of the decisions we made today:
\begin{enumerate}
\item{We will use a spline approach to quantitatively describe the concavity of demographic responses to climate variability. We’re not sure if the spline approach will work well for short time series but we’ll try and modify as needed if we get garbage results. }
\item{We will use annual means for the climate variable(s) rather than trying to do something fancy with a weighted density kernel to quantify time lags. If we go with SPEI (next point), this already has some “history” built into it since we need to specify the time window of climate integration.}
\item{We will use SPEI, a widely applied index of drought stress based on evapotranspirative potential. This really simplifies this. There was some discussion of \textit{also} using straight-up temperature and precipitation, and the code will likely be written generally enough that it should be easy to swap in different climate variables if we decide to go that route. We are not sure if SPEI is super meaningful for marine and/or freshwater taxa but we’ll proceed in treating these data the same as the terrestrial data, and decide later if the project should be limited to land.}
\item{There will be some work to do in deriving climate change forecasts for SPEI at all of our LTER sites, but this will likely come after we have fit the population models and are ready for some forecasting analysis.}
\end{enumerate}

Lastly, here is the GPP paper that Kai mentioned: https://www.nature.com/articles/nature23021. If you know of other, related synthesis papers (Bene had one in mind) please share with the group. 

\section{To-do list, end of fall semester 2018}
\begin{enumerate}

\item{How do we handle multi-spp data sets? Do we fit the full model with species * as.factor(year) interactions? Is there a way to supress intercepts to make the output more interpretable? Decision: we will model multi-spp data sets together with one random effect distribtuion. If it fits poorly in the diagnostics, we can revisit splitting species.}

\item{How do we handle idiosyncratic weirdness with spatial replication levels? We could combine all spatial reps into one combo variable. This would be convenient but we would lose resolution. Decision: YES.}

\item{We need clear and consistent criteria for inclusion. Do we discard very rare species? Decision: FIT EVERYTHING. }

\item{What if the abundance observation is not an observation of the organism, but something that it is correlated with (crab holes)?  We May need to catalog weird data sets, including incomplete taxonomic info. Gaps will be another potential source of weirdness.}

\item{What do we do for data sets with only genus info? We may not know how big a problem this is until we start.}

\item{We need to develop a pipeline for abundance observations of different data types (count, biomass, density, percent cover). THERE WILL BE A PIPELINE TEAM.}

\item{Do we combine data for species that are represented across multiple studies and LTER sites? BOTH WAYS.}

\item{We should code up a simulation study to see if there are any biases in model selection for piecewise regression models with different breakpoints. SIMULATION TEAM.}

\item{Perhaps most significantly, we need to define the environmental variation that we will use to explain inter-annual variation in population growth rates. This is a big can of worms with several issues to consider. DECISION: ALL WAYS? THERE IS A LOT OF ENTHUSIASM FOR CLIMWIN AS A WAY TO UNDERSTAND WHAT IS ACTUALLY DRIVING INDIVIDUAL SPP, BUT IT CAN BE HARD TO COMPARE ACROSS SPECIES, STORY IS LESS COHESIVE IN TERMS OF CLIMATE CHANGE AND KNOWN ELEMENTS OF CLIMATE VARIANCE THAT ARE KNOWN TO INCREASE. ALSO CLIMWIN WILL MAKE THE PIPELINE ANNOYING AND SLOW.}

\begin{enumerate}
\item{Do we handle environmental data differently for terrestrial versus aquatic organisms? IT DEPENDS. IF WE USE CLIMWIN, WE CAN PUT TOGETHER. MAYBE BOTH WAYS, NEED TO THINK MORE ABOUT. THIS IS A TRAIT, SO MAYBE BETTER TO CONNECT TO VARIATION ON THE BACK END.}

\item{Timing of environmental data needs to align with timing of population growth census year.}

\item{Data sources. Presuming we want to work with climate data, should we used LTER meteorological data? Or PRISM data (or some other downscaled data product)? Could use NOAA for aquatic data.}

\item{Do we try to select individual climate variables that are predictive of the data? If so, how?}
\begin{itemize}
\item{ClimWin is one option that has been developed and vetted by others. We could also do our own variable selection.}
\end{itemize}
\item{Alternatively, we could take some dimensionality-reduction approach to multivariate climate data, without trying to find the specific drivers for each species.}
\begin{itemize}
\item{Drought indices like SPEI or Palmer combine info on preciptation and temperature. Probably a good one-size-fits-all approach for plants, may be not as good for higher trophic levels. Is there an index for aquatic systems?}
\item{We could use PCA to reduce climate dimensions, but the top PC may only explain a small fraction of the climate data, and it may not be the part of climate that is most relevant for the organism. IF WE DID PCA, WPOULD WE HAVE ONE MEGA PCA OR LOTS OF MINI PCAS?}
\end{itemize}
\end{enumerate}
\end{enumerate}


\end{document}